\chapter{INTRODUCTION}

In this section, pes planus and pes cavus deformations, which are foot deformation types, are explained, also an overview of the motivation of the study is presented, and finally the content of the thesis is detailed.

\section{PROBLEM DEFINITION AND MOTIVATION}

In our daily life, we often do not think about the foot's functionality. However, the feet play an important role in the integrity of the movement system, as they carry the body. Some scientists even describe feet as "hardworking workers of the body" because of the critical tasks they perform \cite{veli2004shoes}.

Many different deformities are encountered in the foot since birth of the individual such as pes cavus, pes planus, traumatic injuries and neurological lesions. Foot deformities affect gait and cause compensatory changes in other joints. In addition, knee, hip, and waist problems are also seen. Foot deformities disrupt the shoe fit and require special in-shoe orthoses. Due to improper pressure, pressure points cause skin problems ranging from calluses to skin ulcers.

Since it is estimated that about a third of the population has some form of foot deformity, it is crucially important to detect foot deformities. Even though pes planus and pes cavus are not the most common deformations in the population, they are the foot deformations that most affect the productivity of individuals. Therefore, diagnosing and treating pes planus and pes cavus is exceptionally important for society.

Pes planus is described as a decrease in the slope of the medial longitudinal arch. The prevalence of pes planus in the specific age groups varies between $<$1 percent and 28 percent \cite{ccilli2009prevalence, pfeiffer2006prevalence, abdel2006flat, chen2009flatfoot}. On the other hand, pes cavus is described as an increase in the plantar longitudinal arch slope. In addition, pes cavus is relatively less common than pes planus \cite{kharbuja2017prevalence}.

The first objective of the research is to diagnose pes planus and pes cavus remotely via mobile applications. However, remote diagnosis of pes planus and pes cavus is extremely difficult since there is no reference point in foot detection. Therefore, deep learning networks will be used to find the reference (e.g. foot) in images. The second objective of the study is to model the sole of the foot to provide a more accurate preliminary diagnosis. For this purpose, the function of sole of the foot will be calculated by image processing and regression.

Within the framework of this research, answers will be sought to questions such as;

\begin{itemize}
  \item Whether deep learning algorithms are sufficient in foot detection?
  \item Is it possible to create the sole function with image processing?
  \item Is remote detection of pes planus and pes cavus possible and viable with the achieved detection ratio?
\end{itemize}

In the literature, there are many methods for detecting pes planus and pes cavus. These can be divided into two categories: non-anthropometric and anthtropometric. Anthropometric methods require external tools such as a digital radiography system. Although the literature has proven that radiological imaging provides much better detection, however it is not a suitable method for remote detection. Therefore, this work has been carried out with non-anthropometric methods such as Arch Height Index and Rearfoot Angle.

\section{THESIS STRUCTURE}

Chapter \ref{chp:Evaluation of Technology} provides an introduce to the domains of artificial intelligence, machine learning and image processing in detail. Chapter \ref{chp:Feet Deformities} describes the medical treatments and detection methods of pes planus and pes cavus. Approaches to used to detect pes planus and pes cavus are discussed in Chapter \ref{chp:Methodology}. Chapter \ref{chp:Foot Detection & Primary Diagnosis} provides an overview of data collection and detection system first, followed by a discussion of testing and results. Chapter \ref{chp:Back Side Diagnosis} works on completely different pre-diagnosis algorithms and participants. Chapter \ref{chp:Inner Side Diagnosis Improvement}, which includes discusses the improvement of the initial pre-diagnosis algorithm and its results. In the last chapter, (\ref{chp:ConclusionsAndFutureWork}) potential improvements and new concepts for future work are mentioned.

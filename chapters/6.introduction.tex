\chapter{INTRODUCTION}\label{chp:Introduction}

Although not very aware of it, having healthy feet is of critical importance for the individual. Since the feet form the basis of the body in terms of posture, balance and support, any negativity in the foot structure affects the whole body. For example, we may have difficulty performing activities that we always do, such as climbing stairs or walking, due to a foot problem. This affects our quality of life negatively.

Recently, foot diseases have increased between 61\% and 79\% \cite{pita2017flat, menz2010characteristics}. One of these diseases is foot deformation. It is estimated that about a third of the population has some form of foot deformity. Even though pes planus and pes cavus are not the most common deformations in population, they are the foot deformations that most affect the productivity of individuals. Therefore, diagnosing and treating pes planus and pes cavus is extremely important for society. 

Pes planus is described as a decrease in the slope of the medial longitudinal arch. The prevalence of pes planus in the specific age groups varies between $<$1\% and 28\% \cite{ccilli2009prevalence, pfeiffer2006prevalence, abdel2006flat, chen2009flatfoot}. On the other hand, pes cavus is described as an increase in the plantar longitudinal arch slope. In addition, pes cavus is relatively less common than pes planus \cite{kharbuja2017prevalence}. 

In the literature, there are many methods for detecting pes planus and pes cavus, divided into two categories: non-anthropometric and anthropometric. Anthropometric methods require external tools such as a digital radiography system. Although the literature has proven that radiological imaging provides precise detection, it is not a suitable method for remote detection. Therefore, this work has been carried out with non-anthropometric methods such as arch height index and Rearfoot angle.

The first aim of this study is to remotely diagnose pes planus and pes cavus via mobile applications. However, remote diagnosis of pes planus and pes cavus is extremely difficult as there is no reference point. Therefore, deep learning networks will be used to find the reference (e.g. foot) in images. The second aim of the research is to model the sole of the foot to provide a more accurate preliminary diagnosis. For this, it will be tried to obtain results related to image processing and regression.

Within the framework of this research, answers will be sought to questions such as;

\begin{itemize}
  \item Whether deep learning algorithms are sufficient in foot detection?
  \item Is it possible to create the sole function with image processing?
  \item Is remote detection of pes planus and pes cavus possible with the required detection rate?
\end{itemize}

In chapter \ref{chp:Evaluation of Technology} of this work, the details of the overview of artificial intelligence, machine learning, and image processing are briefly explained. The next section (\ref{chp:Feet Deformities}) describes the medical treatments and detection methods of pes planus and pes cavus. Approaches to detecting pes planus and pes cavus are discussed in Chapter \ref{chp:Methodology}. Chapter \ref{chp:Foot Detection & Primary Diagnosis} provides an overview of the primary system first, followed by a discussion of testing and results. Next, chapter \ref{chp:Back Side Diagnosis} works on completely different pre-diagnosis algorithms and participants. Chapter \ref{chp:Inner Side Diagnosis Improvement}, which includes recent studies, discusses the improvement of the initial pre-diagnosis algorithm and its results. In the last section (\ref{chp:ConclusionsAndFutureWork}) potential improvements and new concepts for future work are mentioned.

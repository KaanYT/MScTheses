

\setcounter{equation}{0}

\appendix
%\addcontentsline{toc}{chapter}{\numberline{}Appendix}


%% this part is used to remove table and figure in the appendix from the list of tables and figure so don't delete it
\let\svaddcontentsline\addcontentsline
\renewcommand\addcontentsline[3]{%
  \ifthenelse{\equal{#1}{lof}}{}%
  {\ifthenelse{\equal{#1}{lot}}{}{\svaddcontentsline{#1}{#2}{#3}}}}
  
%%  you can make change after this line %%
  


\chapter{ADDITIONAL COMPARATIVE RESULTS FOR THE MEAN-RISK MODELS}
\label{sec.app1}

The results presented in this part are obtained according to the
problem instance belongs to the Group 1 with random link capacities.

\begin{table}[htbp]
\begin{center}
\caption{Resource2}
\vspace{23pt}
      \begin{tabular}{|c|c|c|c|}
        \hline
           & \textbf{Cores}  & \textbf{Total LUTs}   &  \textbf{LUTs usage (\%)}\\
           \hline
        Ethernet core & 1 & 1476 & 2.07\\
        \hline
        CAs & 3 & 63966  & 90.00\\
        \hline
        Comparator module  & 1 &  284   &  0.4      \\
        \hline
        DRAM controller    & 1 &  376   &  0.53       \\
        \hline
        Main controller    & 1 &  932   &  1,30         \\
        \hline
        Total design & -  & 65442    & 94.93\\
        \hline
      \end{tabular}
\label{tab:table2}
\end{center}
\end{table}


%%\begin{figure}[htbp]
%%\centering
%%\fbox{\includegraphics[width=.8\columnwidth]{figures/DNAhelical.eps}}
%%\caption{Structure of DNA, single stand and double helix.}
%%\label{fig:DNAhelical2}
%%\end{figure}
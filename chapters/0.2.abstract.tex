\begin{abstract}
According to previous research in this domain, about thirty percent of the world population has some kind of foot deformity. Among all the deformities, pes planus produced by a lack of the medial longitudinal arch of the foot and pes cavus produced by an excessively high plantar longitudinal arch are the foot deformities that most adversely affect the community's quality of life. Bearing in mind the above, the proposed solution aims to pre-diagnose pes planus and pes cavus through a mobile phone application using image processing and deep neural networks with the help of traditional deformity recognition methods existing in the literature. To this end, this work is composed of three studies that build upon each other, to improve the foot deformity detection precision. Accordingly, the first prototype is implemented using machine learning techniques to estimate pes planus and pes cavus using footprint approach and tested 34 participants. To benchmark this prototype, an orthopedist is asked to provide key decision-making points manually so that a comparative calculation can be performed. The comparison results of the two  showed that the expert and prototype findings were in agreement with each other more than 90 percent. As a follow up, another deformity classification method is implemented as a second study, which entailed, rearfoot angle and input from the healthcare professionals. This prototype was tested with 9 participants. Within the scope of the second study, data entries were provided by using the prototype by healthcare professionals. The system results were compared with the deformities detected by the healthcare professionals, and it was observed that the results were matched at a rate of 27.7. In the final study, the detection method developed in the first prototype is improved and tested with the test set used in the second study. The results showed that deformities detected by the physician and the system matched 83.3 percent.
\end{abstract}
\chapter{FEET DEFORMITIES}\label{chp:Feet Deformities}

Determination of the genetic basis of human diseases is one of the main goals of molecular genetics. By the publication of the human genome in 2001

\section{Pes Planus}
\section{Pes Cavus}

The structure of DNA and the nucleotide bases that hold this bonded, paired structure as illustrated in Fig.~\ref{fig:DNAhelical}.

\begin{figure}[htbp]
\centering
\fbox{\includegraphics[width=.8\columnwidth]{figures/DNAhelical.eps}}
\caption{Structure of DNA, single stand and double helix.}
\label{fig:DNAhelical}
\end{figure}

The results presented in this part are obtained according to the
problem instance belongs to the Group 1 with random link capacities.

\begin{table}[htbp]
\begin{center}
\caption{Resource2}
\vspace{23pt}
      \begin{tabular}{|c|c|c|c|}
        \hline
           & \textbf{Cores}  & \textbf{Total LUTs}   &  \textbf{LUTs usage (\%)}\\
           \hline
        Ethernet core & 1 & 1476 & 2.07\\
        \hline
        CAs & 3 & 63966  & 90.00\\
        \hline
        Comparator module  & 1 &  284   &  0.4      \\
        \hline
        DRAM controller    & 1 &  376   &  0.53       \\
        \hline
        Main controller    & 1 &  932   &  1,30         \\
        \hline
        Total design & -  & 65442    & 94.93\\
        \hline
      \end{tabular}
\label{tab:table2}
\end{center}
\end{table}

\subsection{Human Genome Project and Sanger Sequencing} \label{sec:genProject}
	
Sequencing method employed in the project.

\subsection{Next Generation Sequencers}

Since completion of the first human genome sequence, demand for cheaper and faster sequencing methods has increased greatly.

The sequencer was employing the Sanger sequencing method of human genome project in 2001 \cite{petitti2005blood,hong2011state,vapnik2013nature,mitsuoka1990role,rennie2001improving,Yang2012,Yang2013}.

Details of the algorithm up to here is given in .



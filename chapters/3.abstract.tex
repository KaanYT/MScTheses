\begin{abstract}
About thirty percent of the world population has some kind of foot deformity, according to previous researches. Among all the deformities, pes planus produced by a lack of the medial longitudinal arch of the foot and pes cavus produced by an excessively high plantar longitudinal arch are the foot deformities that most adversely affect the community's quality of life. In view of the preceding, this research offers a unique mobile pre-diagnosis approach through a mobile phone application using image processing and deep neural networks. This application diagnoses pes planus and pes cavus with the help of traditional deformity recognition methods created in the literature. This research consists of three different studies. Each of these studies is built on the other. In the first study, a prototype was implemented and tested with 34 participants, 22 men and 12 women, with a mean age of 24.06 years. To benchmark our prototype, an orthopedist was asked to identify key decision-making points used to calculate deformity type on a set of foot images collected from participants. Results showed that the expert and prototype findings were in agreement over 90\%. In the second study, a different deformation detection method was developed, and the prototype was tested with 9 participants, 6 men and 3 women with a mean age of 22.67 years. Within the scope of the second study, data entries were provided by using the prototype by healthcare professionals. The system results were compared with the deformities detected by the healthcare professionals, and it was observed that the results were matched at a rate of 27.7. In the third and final study, the detection method developed in the first prototype was improved and compared with the data collected in the second study. It was observed that the deformities detected by the healthcare professionals and the system results matched at a rate of 83.3 \%.
\end{abstract}
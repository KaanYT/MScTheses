\chapter{CONCLUSIONS AND FUTURE WORK}\label{chp:ConclusionsAndFutureWork}

This research aims to develop a mobile pre-diagnostic platform that will assists healthcare professionals in remotely classifying and evaluating possible patients with foot deformities that is namely pes cavus and pes planus. For this purpose, three studies were conducted to accurately detect pes planus and pes cavus. The first of these studis, unlike other studies, was conducted remotely and unsupervised, so the data are not used together with other studies.

The first study focused on the general flow and requirements of the system, and the detection of foot deformation was tested by remotely receiving and processing data throughout the study. Foot deformity determination was made according to the arch height index. Study results have been verified with data provided remotely by the expert. Although the initial research results were positive and resulted in high accuracy, it cannot be said that a guaranteed conclusion was reached about the first study results since the physician made the control with the data provided by the users and did not detect the foot deformation through physical examination. This study showed that pes planus and pes cavus detection is possible using mobile applications. Therefore, second study was conducted to verify the first study’s results by performing deformity detection under the supervision of healthcare professionals.

The second study focused on the acquisition of examination data and results of rearfoot angle based foot deformity detection algorithm. In this study, mobile applications were used by healthcare professionals and not by end users, to obtain supervised data. Some infrastructure changes were made to enable healthcare professionals to use the system as end-user, and as a result, more accurate data was collected. The second study shows that using a rearfoot angle based foot deformity detection algorithm was not possible for mobile based detection because the pre-diagnosis algorithm was susceptible to degree changes in the critical points lines (5 degrees). Therefore, third study was conducted to verify and improve the  arch height index algorithm

The third study focused on improving foot deformity detection. Throughout this study, the data collected by the healthcare professionals in the second study and the detection method in the first study were used. In this study, the results are significantly improved compared to the second study, while reducing the shortcomings of the arch height index introduced in the first study. The third study showed that pre-diagnose pes planus and pes cavus through a mobile phone application using image processing and deep neural networks is possible.

When the results of 3 studies are examined, the first study gives the best result with a success rate of 91.80 percent, but since the first study data were not obtained in a supervised manner, it cannot be compared with other studies. The success rates of the second and third studies are 27.7 percent and 83.3 percent, respectively. Accordingly, the results of the second study reveal the importance of selecting the correct index method for pre-diagnosis.

Although test trials were conducted in various populations, tests were carried out with limited participants, especially in the second and third studies. Therefore, the results in larger populations may be different.

Many different adaptations, experiments and studies could be done in the future. These are, first, that better results can be achieved by collecting more data and moving manual feature extraction to deep learning exercises, and secondly, problems in data collection can be eliminated by placing various controls in end-user applications and finally, the development of detection algorithms based on other pes planus and pes cavus diagnostic methods accepted in the literature.
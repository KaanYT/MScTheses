\begin{turkishabstract}\label{chp:turkishabstract}

Bu alandaki önceki araştırmalara göre, dünya nüfusunun yaklaşık yüzde otuzunda bir çeşit ayak deformitesi vardır. Tüm deformiteler arasında, ayağın medial longitudinal arkının eksikliğinden kaynaklanan pes planus ve aşırı yüksek plantar longitudinal arkın oluşturduğu pes kavus, toplumun yaşam kalitesini en olumsuz etkileyen ayak deformiteleridir. Yukarıdakiler göz önünde bulundurularak önerilen çözüm, literatürde var olan geleneksel deformite tanıma yöntemleri yardımıyla görüntü işleme ve derin sinir ağları kullanılarak bir cep telefonu uygulaması aracılığıyla pes planus ve pes kavus ön teşhisini yapmayı amaçlamaktadır. Bu amaçla, bu çalışma, ayak deformitesi tespit hassasiyetini geliştirmek için birbiri üzerine inşa edilen üç çalışmadan oluşmaktadır. Buna göre, ilk prototip, pes planus ve pes kavusu tahmin etmek için makine öğrenimi teknikleri kullanılarak 34 katılımcı ile test edilmiştir. Bu prototipi kıyaslamak için bir ortopedistten, karşılaştırmalı bir hesaplamanın yapılabilmesi için kilit karar verme noktalarını manuel olarak sağlaması istenir. İkisinin karşılaştırma sonuçları, uzman ve prototip bulgularının yüzde 90'dan fazla birbiriyle uyumlu olduğunu göstermiştir. Bunu takiben, ikinci bir çalışma olarak, arka ayak açısı ve sağlık profesyonellerinden girdi içeren başka bir deformite sınıflandırma yöntemi uygulanmaktadır. Bu prototip 9 katılımcı ile test edilmiştir. İkinci çalışma kapsamında sağlık profesyonelleri tarafından prototip kullanılarak veri girişleri sağlanmıştır. Sistem sonuçları sağlık profesyonelleri tarafından tespit edilen deformiteler ile karşılaştırılmış ve sonuçların 27,7 oranında eşleştiği gözlemlenmiştir. Üçüncü ve son çalışmada ise birinci prototipte geliştirilen algılama yöntemi geliştirilmiş ve ikinci çalışmada kullanılan test seti ile test edilmiştir. Sonuçlar, hekim ve sistem tarafından tespit edilen deformitelerin yüzde 83,3 eşleştiğini göstermiştir.

\end{turkishabstract}
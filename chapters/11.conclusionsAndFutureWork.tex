\chapter{CONCLUSIONS AND FUTURE WORK}\label{chp:ConclusionsAndFutureWork}

This research aims to develop a mobile pre-diagnostic platform that will assist physicians in remotely classifying and evaluating possible patients with foot deformities that is namely pes cavus and pes planus. For this purpose, three studies were conducted to accurately detect pes planus and pes cavus. The first of these studies, unlike other studies, was conducted remotely and unsupervised, so the data are not used together with other studies.

The first study focused on the general flow and requirements of the system, and the detection of foot deformation was tested by remotely receiving and processing data throughout the study. Foot deformity determination was made according to the arch height index. Study results have been verified with data provided remotely by the expert. Although the initial research results were positive and provided a high detection rate, it cannot be said that a guaranteed conclusion was reached about the foot deformations, since the physician made the control with the data provided by the users and did not physically detect the foot deformation.

The second study focused on the acquisition of examination data and different types of foot deformity detection, and in this study, experts used the developed system as an end user to obtain supervised data. Some infrastructure changes were made to enable experts to use the system as an end user, and as a result, more accurate data was collected. In addition, a different fixation method based on the rearfoot angle has been tried for the detection of foot deformity. This detection method has reduced the detection rate because of its high sensitivity to error.

Recent work has focused on improving foot deformity detection. Throughout this study, the data collected by the experts in the second study and the detection method in the first study were used. The last study mitigates the shortcomings of the arch height index introduced in the first study. Therefore, the results were considerably improved compared to the second study.

When the results of 3 studies are examined, the first study gives the best result with a success rate of 91.80 percent, but since the first study data were not obtained in a supervised manner, it cannot be compared with other studies. The success rates of the second and third studies are 27.7 percent and 83.3 percent, respectively. Accordingly, the results of the second study reveal the importance of selecting the index method.

Although test trials were conducted in various populations, tests were carried out with limited participants, especially in the second and third studies. Therefore, the results in larger populations may be different.

Many different adaptations, experiments and studies could be done in the future. These are, first, that better results can be achieved by collecting more data and moving manual feature extraction to deep learning exercises, and secondly, problems in data collection can be eliminated by placing various controls in end-user applications and finally, the development of detection algorithms based on other pes planus and pes cavus diagnostic methods accepted in the literature.
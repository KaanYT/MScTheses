\chapter{CONCLUSIONS AND FUTURE WORK}\label{chp:ConclusionsAndFutureWork}

This research aims to develop a mobile pre-diagnostic platform that will assist physicians in remotely classifying and evaluating possible patients with foot deformities, specifically pes cavus and pes planus. Therefore, three studies were conducted to accurately detect pes planus and pes cavus. However, unlike the other two, the first research was conducted remotely and unsupervised, and for this reason, the data are not used with other studies.

Initial research focused on the overall flow and requirements of the system. In this study, the detection of foot deformation was tested by remotely receiving and processing these data. In this study, results were validated by the data provided by the specialist remotely. Initial research results were auspicious, but performing them physically and comparing detailed results is necessary. In addition, foot deformity detection was based on the arch height index. 

The second study focused on getting examination data and different foot deformity detection types. In the second study, specialists used the system to get supervised data. In order to achieve this, some infrastructure changes have been made. As a result, more error-free data was collected. In addition, foot deformity detection is changed based on the rearfoot angle. This change increased the error due to sensitivity.

The final study concentrated on improving foot deformity detection. In addition, the final study utilized the first research detection algorithm to improve specialists' data results collected in the second study. The final study reduces the arch height index's shortcomings introduced in the first study. Therefore results were highly improved compared to the second study. 

Since the first study data were different, it is not comparable with the other studies. However, the success rate of 91.80\% reveals the importance of the first study. On the other hand, success rates were 27,7\% and 83,3\% in the second and third studies, respectively. Therefore, the second study's results reveal the importance of selecting the index method.

Although test trials were conducted in various populations, tests were carried out with limited participants, especially in the second and third studies. 
Therefore, the results in larger populations may be different.

Due to a shortage of time, numerous different adaptations, experiments, and studies have been postponed. Therefore some of the possible future works will be discussed. Firstly, better results can be obtained by collecting more data and moving the manual feature extraction process to deep learning exercises. Secondly, problems in data collection can be eliminated by placing various controls on end-user applications. Finally, detection algorithms based on other pes planus and pes cavus diagnosis methods accepted in the literature can be developed.


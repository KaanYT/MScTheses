\begin{ozet}
Daha önceki araştırmalara göre dünya nüfusunun yaklaşık yüzde otuzunda bir çeşit ayak deformitesi vardır. Tüm deformiteler arasında, ayağın medial longitudinal arkının eksikliğinden kaynaklanan pes planus ve aşırı yüksek plantar longitudinal arkın oluşturduğu pes cavus, toplumun yaşam kalitesini en olumsuz etkileyen ayak deformiteleridir. Öncekiler ışığında, bu araştırma, görüntü işleme ve derin sinir ağlarını kullanan bir cep telefonu uygulaması aracılığıyla benzersiz bir mobil ön tanı yaklaşımı sunmaktadır. Bu uygulama, literatürde oluşturulmuş geleneksel deformite tanıma yöntemleri yardımıyla pes planus ve pes kavus tanısını koyar. Araştırma her biri diğerinin üzerine inşa edilmiş üç farklı çalışmadan oluşur. İlk çalışmada, yaş ortalaması 24.06 olan 22 erkek ve 12 kadın olmak üzere 34 katılımcı ile bir prototip uygulanmış ve test edilmiştir. Prototipimizi kıyaslamak için bir ortopedistten, katılımcılardan toplanan bir dizi ayak görüntüsü üzerinde deformite tipini hesaplamak için kullanılan temel karar verme noktalarını belirlemesi istenmiştir. Sonuçlar, uzman ve prototip bulgularının \%90'ın üzerinde uyum içinde olduğunu göstermiştir. İkinci çalışmada ise farklı bir deformasyon tespit yöntemi geliştirilmiş ve prototip, yaş ortalaması 22,67 yıl olan 6 erkek ve 3 kadın olmak üzere 9 katılımcı ile test edilmiştir. İkinci çalışma kapsamında sağlık profesyonelleri tarafından prototip kullanılarak veri girişleri sağlanmıştır. Sistem sonuçları sağlık profesyonelleri tarafından tespit edilen deformiteler ile karşılaştırılmış ve sonuçların 27,7 oranında eşleştiği gözlemlenmiştir. Üçüncü ve son çalışmada, birinci prototipte geliştirilen tespit yöntemi geliştirilmiş ve ikinci çalışmada toplanan verilerle karşılaştırılmıştır. Sağlık profesyonelleri tarafından tespit edilen deformiteler ile sistem sonuçlarının 83,3\% oranında eşleştiği görülmüştür.
\end{ozet}
